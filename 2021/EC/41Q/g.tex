\iffalse
\let\negmedspace\undefined
\let\negthickspace\undefined
\documentclass[journal,12pt,twocolumn]{IEEEtran}
\usepackage{pgfplots}
\pgfplotsset{compat=1.17}
\usepackage{cite}
\usepackage{amsmath,amssymb,amsfonts,amsthm}
\usepackage{algorithmic}
\usepackage{graphicx}
\usepackage{textcomp}
\usepackage{xcolor}
\usepackage{txfonts}
\usepackage{listings}
\usepackage{enumitem}
\usepackage{mathtools}
\usepackage{gensymb}
\usepackage{comment}
\usepackage[breaklinks=true]{hyperref}
\usepackage{tkz-euclide} 
\usepackage{listings}
\usepackage{gvv}                                        
\def\inputGnumericTable{}                                 
\usepackage[latin1]{inputenc}                                
\usepackage{color}                                            
\usepackage{array}                                            
\usepackage{longtable}                                       
\usepackage{calc}                                             
\usepackage{multirow}                                         
\usepackage{hhline}                                           
\usepackage{ifthen}                                           
\usepackage{lscape}
\newtheorem{theorem}{Theorem}[section]
\newtheorem{problem}{Problem}
\newtheorem{proposition}{Proposition}[section]
\newtheorem{lemma}{Lemma}[section]
\newtheorem{corollary}[theorem]{Corollary}
\newtheorem{example}{Example}[section]
\newtheorem{definition}[problem]{Definition}
\newcommand{\BEQA}{\begin{eqnarray}}
\newcommand{\EEQA}{\end{eqnarray}}
\newcommand{\define}{\stackrel{\triangle}{=}}
\theoremstyle{remark}
\newtheorem{rem}{Remark}
\begin{document}
\bibliographystyle{IEEEtran}
\vspace{3cm}
\title{GATE EC 41Q}
\author{EE23BTECH11021 - GANNE GOPI CHANDU$^{*}$% <-this % stops a space
}
\maketitle
\bigskip
\renewcommand{\thefigure}{\theenumi}
\renewcommand{\thetable}{\theenumi}
\bibliographystyle{IEEEtran}
\textbf{Question}\\
Consider the signals \(x[n] = 2^{n-1} u[-n+2]\) and \(y[n] = 2^{-n+2} u[n+1]\), where \(u[n]\) is the unit step sequence. Let \(X(e^{j\omega})\) and \(Y(e^{j\omega})\) be the discrete-time Fourier transform of \(x[n]\) and \(y[n]\), respectively. The value of the integral
\[
\frac{1}{2\pi} \int_{0}^{2\pi} X(e^{j\omega}) Y(e^{-j\omega}) d\omega
\]
(rounded off to one decimal place) is.\\
\textbf{Solution}\\
\fi
\begin{table}[!h]
\begin{center}
\renewcommand\thetable{1}
\begin{tabular}{ |c|c|c| } 
  \hline
    Symbol & Value & description \\ 
  \hline
  $x[n] $ & $2^{n-1}u[-n+2]$ & Discrete time signal  \\ 
  \hline
  $y[n] $ & $2^{-n+2}u[n+1]$ & Discrete time signal  \\ 
  \hline
\end{tabular}
\end{center}
\caption{}
\end{table}
\begin{align}
     x[n]*y[n] && \xleftrightarrow[transform]{Fourier} && X(e^{j\omega}) Y(e^{j\omega})\\
 x[n] && \xleftrightarrow [transform]{Fourier} && X(e^{j\omega}) \\
 y[n] && \xleftrightarrow [transform]{Fourier} && Y(e^{j\omega}) 
\end{align}
The
 \begin{align}
       y(n) && \xleftrightarrow [transform]{Fourier} && y(e^{j\omega})
\end{align}
By using the time reversal property:
\begin{align}
y[-n] && \xleftrightarrow [transform]{Fourier} && y(e^{-j\omega})
\end{align}
Let assume
\begin{align}
     z[n]& = x[n] * y[-n]\\
     Z\brak{e^{j \omega}} &=X(e^{j\omega}) Y(e^{-j\omega})
 \end{align}
\begin{align}
      z[n]& =\frac{1}{2\pi} \int_{0}^{2\pi} Z(e^{j\omega})e^{j \omega n} d\omega \\
      &=\frac{1}{2\pi} \int_{0}^{2\pi}  X(e^{j\omega}) Y(e^{-j\omega})e^{j \omega n} d\omega.
 \end{align}
 putting  n=0, we get
\begin{align}
    z[0]&=\frac{1}{2\pi} \int_{0}^{2\pi} X(e^{j\omega}) Y(e^{-j\omega}) d\omega
\end{align}
\begin{align}
    z[n] = x[n] * y[-n]\\
 &= \sum_{k=-\infty}^{\infty} 2^{k-1} u[-k+2]\cdot 2^{n-k+2} u[-n+k+1]\\
 &= \sum_{k=-\infty}^{2} 2^{k-1} \cdot 2^{n-k+2} u[-n+k+1]\\
 &= \sum_{k=-\infty}^{2} 2^{k-1+n-k+2} u[-n+k+1]\\
 &= \sum_{k=-\infty}^{2} 2^{n+1} u[-n+k+1]
\end{align}

Putting $n = 0$, we get:
\begin{align}
     \frac{1}{2\pi} \int_{0}^{2\pi} X(e^{j\omega}) Y(e^{-j\omega}) d\omega &= z[0]\\
     &= \sum_{k=-\infty}^{2} 2 \cdot u[k+1] \\
     &=\sum_{k=-1}^{2} 2(1) = 2 \times 4 \\
     &= 8
\end{align}

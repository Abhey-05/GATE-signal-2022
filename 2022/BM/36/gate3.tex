\iffalse
\let\negmedspace\undefined
\let\negthickspace\undefined
\documentclass[journal,12pt,twocolumn]{IEEEtran}
\usepackage{cite}
\usepackage{amsmath,amssymb,amsfonts,amsthm}
\usepackage{algorithmic}
\usepackage{graphicx}
\usepackage{textcomp}
\usepackage{xcolor}
\usepackage{txfonts}
\usepackage{listings}
\usepackage{enumitem}
\usepackage{mathtools}
\usepackage{gensymb}
\usepackage{comment}
\usepackage[breaklinks=true]{hyperref}
\usepackage{tkz-euclide} 
\usepackage{listings}
\usepackage{gvv}                                        
\def\inputGnumericTable{}                                 
\usepackage[latin1]{inputenc}                                
\usepackage{color}                                            
\usepackage{array}                                            
\usepackage{longtable}                                       
\usepackage{calc}                                             
\usepackage{multirow}                                         
\usepackage{hhline}                                           
\usepackage{ifthen}                                           
\usepackage{lscape}

\newtheorem{theorem}{Theorem}[section]
\newtheorem{problem}{Problem}
\newtheorem{proposition}{Proposition}[section]
\newtheorem{lemma}{Lemma}[section]
\newtheorem{corollary}[theorem]{Corollary}
\newtheorem{example}{Example}[section]
\newtheorem{definition}[problem]{Definition}
\newcommand{\BEQA}{\begin{eqnarray}}
\newcommand{\EEQA}{\end{eqnarray}}
\newcommand{\define}{\stackrel{\triangle}{=}}
\theoremstyle{remark}
\newtheorem{rem}{Remark}
\begin{document}
\parindent 0px
\bibliographystyle{IEEEtran}
\title{GATE: BM - 36.2022}
\author{EE22BTECH11219 - Rada Sai Sujan$^{}$% <-this % stops a space
}
\maketitle
\newpage
\bigskip
\section*{Question}
In the complex $z$-domain, the value of integral $\oint_{C}\frac{z^3-9}{3z-i}\;dz$ is   \\
\begin{enumerate}[label=(\alph*)]
    \item $\frac{2\pi}{81}-6i\pi$ 
    \item $\frac{2\pi}{81}+6i\pi$ 
    \item $-\frac{2\pi}{81}+6i\pi$ 
    \item $-\frac{2\pi}{81}-6i\pi$ 
\end{enumerate} \hfill(GATE 2022 BM)    \\
\solution
\fi

Simplyfying the Contour Integral to the standard form we get,
\begin{align}
    \oint_{C}\frac{z^3-9}{3z-i}\;dz &= \frac{1}{3}\oint_{C}\frac{z^3-9}{z-\frac{i}{3}}\;dz
\end{align}
From Cauchy's residue theorem,
\begin{align}
    \oint_{C}f(z)\;dz &= 2\pi i\sum R_j \label{equation:bm.2022.36Q.2}
\end{align}
We can observe a non-repeated pole at $z=\frac{i}{3}$ and thus $a=\frac{i}{3}$,
\begin{align}
    R &= \lim\limits_{z\to a}\brak{z-a}f\brak{z}    \\
    \implies R &= \frac{1}{3}\lim\limits_{z\to \frac{i}{3}}\brak{z-\frac{i}{3}}\frac{z^3-9}{z-\frac{i}{3}}  \\
    &= \frac{-i}{81}-3  \label{equation:bm.2022.36Q.5}
\end{align}
Therefore, from \eqref{equation:bm.2022.36Q.2} and \eqref{equation:bm.2022.36Q.5}
\begin{align}
    \oint_{C}\frac{z^3-9}{3z-i}\;dz &= \frac{2\pi}{81}-6i\pi
\end{align}

% \iffalse
\let\negmedspace\undefined
\let\negthickspace\undefined
\documentclass[journal,12pt,twocolumn]{IEEEtran}
\usepackage{cite}
\usepackage{amsmath,amssymb,amsfonts,amsthm}
\usepackage{algorithmic}
\usepackage{graphicx}
\usepackage{textcomp}
\usepackage{xcolor}
\usepackage{pgfplots}
\usepackage{txfonts}
\usepackage{listings}
\usepackage{enumitem}
\usepackage{mathtools}
\usepackage{gensymb}
\usepackage{comment}
\usepackage[breaklinks=true]{hyperref}
\usepackage{tkz-euclide} 
\usepackage{listings}
\usepackage{gvv}                                        
\def\inputGnumericTable{}                                 
\usepackage[latin1]{inputenc}                                
\usepackage{color}                                            
\usepackage{array}                                            
\usepackage{longtable}                                       
\usepackage{calc}                                             
\usepackage{multirow}                                         
\usepackage{hhline}                                           
\usepackage{ifthen}                                           
\usepackage{lscape}

\newtheorem{theorem}{Theorem}[section]
\newtheorem{problem}{Problem}
\newtheorem{proposition}{Proposition}[section]
\newtheorem{lemma}{Lemma}[section]
\newtheorem{corollary}[theorem]{Corollary}
\newtheorem{example}{Example}[section]
\newtheorem{definition}[problem]{Definition}
\newcommand{\BEQA}{\begin{eqnarray}}
\newcommand{\EEQA}{\end{eqnarray}}
\newcommand{\define}{\stackrel{\triangle}{=}}
\theoremstyle{remark}
\newtheorem{rem}{Remark}
\begin{document}
\parindent 0px
\bibliographystyle{IEEEtran}
\title{GATE: ME - 14.2022}
\author{EE22BTECH11219 - Rada Sai Sujan$^{}$% <-this % stops a space
}
\maketitle
\newpage
\bigskip
\section*{Question}
The fourier series expansion of $x^3$ in the interval $-1\leq x\leq 1$with periodic continuation has
\begin{enumerate}[label=(\alph*)]
    \item only sine terms
    \item only cosine terms
    \item both sine and cosine terms
    \item only sine terms and a non-zero constant
\end{enumerate} \hfill(GATE 2022 ME)    \\
\solution

Fourier series expansion of the function $x\brak{t}$ in the interval $[-L,L]$ can be given by: \\
\begin{align}
    x\brak{t}=a_0+\sum\limits_{n=1}^{\infty}a_n\cos\brak{\frac{n\pi t}{L}}+\sum\limits_{n=1}^{\infty}b_n\sin\brak{\frac{n\pi t}{L}}
\end{align}
where,
\begin{align}
    a_0&=\frac{1}{2L}\int\limits_{-L}^{L}f\brak{t}\,dt  \\
    a_n&=\frac{1}{2L}\int\limits_{-L}^{L}f\brak{t}\cos\brak{\frac{n\pi t}{L}}\,dt  \\
    b_n&=\frac{1}{2L}\int\limits_{-L}^{L}f\brak{t}\sin\brak{\frac{n\pi t}{L}}\,dt  \\
\end{align}
Therefore, the expansion can be given by:
\begin{align}
    t^3=a_0+\sum\limits_{n=1}^{\infty}a_n\cos\brak{n\pi t}+\sum\limits_{n=1}^{\infty}b_n\sin\brak{n\pi t}
\end{align}
Since $t^3$ is an odd function,
\begin{align}
    a_0&=a_n=0   \\
    b_n&=\frac{1}{2}\int\limits_{-1}^{1}t^3\sin\brak{n\pi t}\,dt    \\
    &=\brak{-1}^{n+1}\brak{\frac{2}{n\pi}-\frac{12}{\brak{n\pi}^3}} \
\end{align}
\begin{align}
    \implies t^3&=\sum\limits_{n=1}^{\infty}b_n\sin\brak{\frac{n\pi t}{L}}
\end{align}
$\therefore$It contains only sine terms.
\end{document}

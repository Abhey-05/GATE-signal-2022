\iffalse
\let\negmedspace\undefined
\let\negthickspace\undefined
\documentclass[journal,12pt,twocolumn]{IEEEtran}
\usepackage{cite}
\usepackage{amsmath,amssymb,amsfonts,amsthm}
\usepackage{algorithmic}
\usepackage{graphicx}
\usepackage{textcomp}
\usepackage{xcolor}
\usepackage{txfonts}
\usepackage{listings}
\usepackage{enumitem}
\usepackage{mathtools}
\usepackage{gensymb}
\usepackage{comment}
\usepackage[breaklinks=true]{hyperref}
\usepackage{tkz-euclide} 
\usepackage{listings}
\usepackage{gvv}                                        
\def\inputGnumericTable{}                                 
\usepackage[latin1]{inputenc}                                
\usepackage{color}                                            
\usepackage{array}                                            
\usepackage{longtable}                                       
\usepackage{calc}                                             
\usepackage{multirow}                                         
\usepackage{hhline}                                           
\usepackage{ifthen}                                           
\usepackage{lscape}

\makeatletter

\newcommand*{\underarrow}{\def\@underarrow{\relax}\@ifstar{\@@underarrow}{\def\@underarrow{\hidewidth}\@@underarrow}}
\newcommand*{\@@underarrow}[2][]{\underset{\@underarrow\substack{\uparrow\if\relax\detokenize{#1}\relax\else\\#1\fi}\@underarrow}{#2}}

\newcommand*{\overarrow}{\def\@overarrow{\relax}\@ifstar{\@@overarrow}{\def\@overarrow{\hidewidth}\@@overarrow}}
\newcommand*{\@@overarrow}[2][]{\overset{\@overarrow\substack{\if\relax\detokenize{#1}\relax\else#1\\\fi\downarrow}\@overarrow}{#2}}
\makeatother
\newtheorem{theorem}{Theorem}[section]
\newtheorem{problem}{Problem}
\newtheorem{proposition}{Proposition}[section]
\newtheorem{lemma}{Lemma}[section]
\newtheorem{corollary}[theorem]{Corollary}
\newtheorem{example}{Example}[section]
\newtheorem{definition}[problem]{Definition}
\newcommand{\BEQA}{\begin{eqnarray}}
\newcommand{\EEQA}{\end{eqnarray}}
\newcommand{\define}{\stackrel{\triangle}{=}}
\theoremstyle{remark}
\newtheorem{rem}{Remark}
\begin{document}
\parindent 0px

\bibliographystyle{IEEEtran}
\vspace{3cm}

\title{Assignment\\[1ex]GATE-EE-50}
\author{EE23BTECH11034 - Prabhat Kukunuri$^{}$% <-this % stops a space
}
\maketitle
\newpage
\bigskip

\renewcommand{\thefigure}{\theenumi}
\renewcommand{\thetable}{\theenumi}
\section{Question}
The Fourier transform X\brak{j\omega} of the signal\\ $x(t)=\frac{t}{\brak{1+t^2}^2}$ is \rule{1.5cm}{0.15mm}.
\begin{enumerate}
	\item[(A)] $\frac{\pi}{2j}\omega e^{-\abs{\omega}}$
	\item[(B)] $\frac{\pi}{2}\omega e^{-\abs{\omega}}$
	\item[(C)] $\frac{\pi}{2j}e^{-\abs{\omega}}$
	\item[(D)] $\frac{\pi}{2}e^{-\abs{\omega}}$
\end{enumerate}
\solution
\fi
\begin{table}[h]
    \centering
    \begin{tabular}{|p{2cm}|p{2.80cm}|p{2.70cm}|}
    \hline
    Symbol&Value&Description\\ \hline
    $$x(t)$$&$$\frac{t}{\brak{1+t^2}^2}$$&$$\text{Signal}$$\\\hline
    $$X\brak{\omega}$$&$$\int_{t=-\infty}^{\infty}x\brak{t}e^{-j\omega t}dt$$& Fourier transform of $x\brak{t}$\\\hline
    \end{tabular}
    \caption{Variable description}
    \label{tab:GATE-2022-EC-15-1}
\end{table}\\
The Fourier transform of the form x\brak{t}=$e^{-a\abs{t}}$ is 
\begin{align}
    x\brak{t}&\xleftrightarrow{\text{F.T}} X\brak{\omega}\\
    X\brak{\omega}&= \frac{2a}{a^2+\omega^2}
\end{align}
Consider, 
\begin{align}
    x\brak{t}&=e^{-\abs{t}}\\
    X\brak{\omega}&=\frac{2}{1+\omega^2}
\end{align}
By using differentiation property from \eqref{eq:Differentiation-property},
\begin{align}
    tx\brak{t}&\xleftrightarrow{\text{F.T}}j\frac{d}{d\omega}X\brak{\omega}\\
     tx\brak{t}&\xleftrightarrow{\text{F.T}}j\sbrak{\frac{d}{d\omega}\brak{\frac{2}{1+\omega^2}}}\\
     te^{-\abs{t}}&\xleftrightarrow{\text{F.T}}\frac{-4j\omega}{\brak{1+\omega^2}^2}
\end{align}
Applying duality property from \eqref{eq:Duality-property},
\begin{align}
    \frac{-4jt}{\brak{1+t^2}^2}&\xleftrightarrow{\text{F.T}}2\pi\brak{-\omega}e^{-\abs{-\omega}}\\
    \frac{t}{\brak{1+t^2}^2}&\xleftrightarrow{\text{F.T}}\frac{-2\pi\omega e^{-\abs{\omega}}}{-4j}\\
    \frac{t}{\brak{1+t^2}^2}&\xleftrightarrow{\text{F.T}}\frac{\pi}{2j}\omega e^{-\abs{\omega}}
\end{align}
%\end{document}

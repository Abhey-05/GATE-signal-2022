\iffalse
\documentclass[journal,12pt,twocolumn]{IEEEtran}
\usepackage{amsmath,amssymb,amsfonts,amsthm}
\usepackage{txfonts}
\usepackage{tkz-euclide} 
\usepackage{listings}
\usepackage{gvv}       
\usepackage[latin1]{inputenc}   
\usepackage{array}  
\usepackage{tikz}
\usepackage{circuitikz}

\begin{document}

\bibliographystyle{IEEEtran}

\vspace{3cm}

\title{}
\author{EE23BTECH11217 - Prajwal M$^{*}$
}
\maketitle
\newpage
\bigskip

\renewcommand{\thefigure}{\theenumi}
\renewcommand{\thetable}{\theenumi}

\section*{EE 16}
The steady state output $V_{out}$ of the circuit shown below, will
\begin{figure}[h]
    \centering
    \begin{circuitikz}[american voltages]
    \draw (0,0) node[op amp] (opamp) {};
    \draw (opamp.+) node[above]{$v_{+}$} to (-2,-0.5);
    \draw (opamp.-) node[above]{$v_{-}$} to (-2, 0.5);
    \draw (opamp.out) to (2, 0)node[right]{$v_{out}$};
    \draw (opamp.down) to (-0.1, -1) node[below]{$-v_{EE}$};
    \draw (opamp.up) to (-0.1, 1)node[above]{$+v_{DD}$};
    \draw (-2,0.5) to [R, l_=$R_1$](-3,0.5) to (-3.5, 0.5) to [V, l_=$0.1v$] (-3.5, -2) node[ground]{};
    \draw (-2, -0.5) to [R, l=$R_2$] (-2, -2) node[ground]{};
    \draw (-1.5,0.5) to (-1.5, 2) to [C, l=$C_1$] (1.5, 2) to (1.5, 0);
\end{circuitikz}

    \caption{circuit}
    \label{fig: 217.EE.16.1}
\end{figure}

\begin{enumerate}
    \item saturate to $+V_{DD}$
    \item saturate to $-V_{EE}$
    \item become equal to $0.1V$
    \item become equal to $-0.1V$
\end{enumerate}

\noindent Solution: \\

\fi
\begin{table}[h]
    \centering
    \begin{tabular}{|c|c|}
\hline
    Parameters & Description \\\hline
    $v_{\text{out}}$  & Steady State Output Voltage  \\\hline
    $V_{\text{out}}$ & Laplace Transform of $v_{\text{out}}$\\\hline
\end{tabular}

    \caption{Parameter description}
\label{tab: 217.EE.16.1}
\end{table}

\begin{figure}[h]
    \centering
    \begin{circuitikz}[american voltages]
    \draw (0,0) node[op amp] (opamp) {};
    \draw (opamp.+) node[above]{$V_{+}$} to (-2,-0.5);
    \draw (opamp.-) node[above]{$V_{-}$}to (-2, 0.5);
    \draw (opamp.out) to (2, 0)node[right]{$V_{out}$};
    \draw (opamp.down) to (-0.1, -1) node[below]{$-V_{EE}$};
    \draw (opamp.up) to (-0.1, 1)node[above]{$+V_{DD}$};
    \draw (-2,0.5) to [R, l_=$R_1$](-3,0.5) to (-3.5, 0.5) to [V, l_=$0.1V$] (-3.5, -2) node[ground]{};
    \draw (-2, -0.5) to [R, l=$R_2$] (-2, -2) node[ground]{};
    \draw (-1.5,0.5) to (-1.5, 2) to [R, l=$\frac{1}{sC_1}$] (1.5, 2) to (1.5, 0);
\end{circuitikz}

    \caption{s-domain circuit}
    \label{fig: 217.EE.16.2}
\end{figure}

for an ideal OP amp,
\begin{align}
    V_+ & = 0V \\
    V_- & = 0V\label{217.EE.16.1}
\end{align}
using KVL,
\begin{align}
    0 & = \frac{V_{-} - 0.1}{R_1} + C_1 s\brak{V_{-} - V_{out}}\\
    V_{out} & = \frac{V_- -0.1}{R_1C_1s} + V_-\\
    & = -\frac{0.1}{R_1C_1s} & \text{using \eqref{217.EE.16.1}}\\
    V_{out} & \system{L^-} v_{out}\\
    v_{out} & = -\frac{0.1}{R_1C_1}t\\
    v_{out} & = max\cbrak{-v_{EE}, -\frac{1}{R_1C_1}t}
\end{align}

Hence, $v_{out}$ saturates to $-v_{EE}$

\iffalse
\let\negmedspace\undefined
\let\negthickspace\undefined
\documentclass[journal,12pt,twocolumn]{IEEEtran}
\usepackage{cite}
\usepackage{amsmath,amssymb,amsfonts,amsthm}
\usepackage{algorithmic}
\usepackage{graphicx}
\usepackage{textcomp}
\usepackage{xcolor}
\usepackage{txfonts}
\usepackage{listings}
\usepackage{enumitem}
\usepackage{mathtools}
\usepackage{gensymb}
\usepackage{comment}
\usepackage[breaklinks=true]{hyperref}
\usepackage{tkz-euclide} 
\usepackage{listings}
\usepackage{gvv}                                        
\def\inputGnumericTable{}                                 
\usepackage[latin1]{inputenc}                                
\usepackage{color}                                            
\usepackage{array}                                            
\usepackage{longtable}                                       
\usepackage{calc}                                             
\usepackage{multirow}                                         
\usepackage{hhline}                                           
\usepackage{ifthen}                                           
\usepackage{lscape}
\newtheorem{theorem}{Theorem}[section]
\newtheorem{problem}{Problem}
\newtheorem{proposition}{Proposition}[section]
\newtheorem{lemma}{Lemma}[section]
\newtheorem{corollary}[theorem]{Corollary}
\newtheorem{example}{Example}[section]
\newtheorem{definition}[problem]{Definition}
\newcommand{\BEQA}{\begin{eqnarray}}
\newcommand{\EEQA}{\end{eqnarray}}
\newcommand{\define}{\stackrel{\triangle}{=}}
\theoremstyle{remark}
\usepackage{circuitikz}
\newtheorem{rem}{Remark}
\begin{document}
\parindent 0px

\bibliographystyle{IEEEtran}
\vspace{3cm}

\title{Assignment\\[1ex]GATE-EE-33}
\author{EE23BTECH11034 - Prabhat Kukunuri$^{}$% <-this % stops a space
}
\maketitle
\newpage
\bigskip

\renewcommand{\thefigure}{\theenumi}
\renewcommand{\thetable}{\theenumi}
\section{Question}
The voltage at the input of an AC-DC rectifier is given by $v\brak{t}=230\sqrt{2}\sin{\omega t}$, where $\omega=2\pi\times 50$rad/s. The input current drawn by the rectifier is given by
\begin{align*}
    i\brak{t}=10\sin\brak{\omega t-\frac{\pi}{3}}+4\sin\brak{3\omega t-\frac{\pi}{6}}+3\sin\brak{5\omega t-\frac{\pi}{3}}
\end{align*}
The power input, (rounded off to two decimal places), is\rule{1.5cm}{0.15mm}lag.


\solution\\
\fi
\begin{table}[h]
    \centering
    \begin{tabular}{|p{2cm}|p{2.80cm}|p{2.70cm}|}
    \hline
    Symbol&Value&Description\\ \hline
    $$v(t)$$&$$230\sqrt{2}\sin{\omega t}$$&$$\text{input voltage}$$\\ \hline
    $$\omega$$&$$100\pi rad/s$$&$$\text{Angular velocity}$$\\ \hline
    $$\text{Power Factor}$$&$$\frac{P_{avg}}{V_{rms}I_{rms}}$$&$$-$$\\\hline
    $$\cos{\varphi}$$&$$\frac{1}{2}$$&Fundamental displacement factor\\ \hline
    $$\varphi$$&$$\frac{\pi}{3}$$&angle between v\brak{t} and $I_n$\\ \hline
    $$I_n$$&$$10\sin\brak{\omega t-\frac{\pi}{3}}$$&fundamental component of current\\\hline
\end{tabular}
    \caption{Variable description}
    \label{tab:GATE.2022.EE.33.1}
\end{table}\\
\begin{align}
  P_{avg}=\frac{1}{T}\int_{0}^{T}V\brak{t}I\brak{t}dt
\end{align}
For current sources of the form $I\brak{t}$=$I_0$+$I_1\brak{t}$+...+$I_n\brak{t}$
\begin{align}
    P_{avg}=&\frac{1}{T}\sum_{1}^{n}\int_{0}^{T}V\brak{t}I_{n}\brak{t}dt\\
    P_{avg}=&\frac{1}{T}\sum_{1}^{n}\int_{0}^{T}V_{pk}sin\brak{\omega t}I_{pk\brak{n}}\sin\brak{\omega t+\varphi}dt\\
    P_{avg}=&\sum_{0}^{n}\frac{v_{pk}I_{\brak{n}pk}}{2}\cos{\varphi}
\end{align}
For a sine wave signal $V_{pk}=V_{rms}\sqrt{2}$
\begin{align}
    P_{avg}&=\sum_{0}^{n}\brak{v_{rms}}\brak{I_{\brak{n}rms}}\cos{\varphi}\\
    &\text{Power Factor}=\frac{\sum_{0}^{n}I_{\brak{n}rms}\cos{\varphi}}{I_{rms}}\\
    I_{rms}&=\sqrt{\brak{\frac{10}{\sqrt{2}}}^2+\brak{\frac{4}{\sqrt{2}}^2}+\brak{\frac{3}{\sqrt{2}}}^2}= 7.905A
\end{align}
The rms value of fundamental value of current 
\begin{align}
\brak{I_{\brak{1}rms}}=\sqrt{\brak{\frac{10}{\sqrt{2}}}^2}\\
\varphi=30^{^\circ}
\end{align}
\begin{align}
    \text{Power Factor}&=\frac{\frac{10}{\sqrt{2}}\cos{30}}{7.905}\\
    \implies&0.4473
\end{align}
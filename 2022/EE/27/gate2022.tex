\iffalse
\let\negmedspace\undefined
\let\negthickspace\undefined
\documentclass[journal,12pt,twocolumn]{IEEEtran}
\usepackage{cite}
\usepackage{amsmath,amssymb,amsfonts,amsthm}
\usepackage{algorithmic}
\usepackage{graphicx}
\usepackage{textcomp}
\usepackage{xcolor}
\usepackage{txfonts}
\usepackage{listings}
\usepackage{enumitem}
\usepackage{mathtools}
\usepackage{gensymb}
\usepackage{comment}
\usepackage[breaklinks=true]{hyperref}
\usepackage{tkz-euclide} 
\usepackage{listings}                                   
\def\inputGnumericTable{}                                 
\usepackage[latin1]{inputenc}                                
\usepackage{color}                                            
\usepackage{array}                                            
\usepackage{longtable}                                       
\usepackage{calc}  
\usepackage{circuitikz}                                           
\usepackage{multirow}                                         
\usepackage{hhline}                                           
\usepackage{ifthen}                                           
\usepackage{lscape}
\newtheorem{theorem}{Theorem}[section]
\newtheorem{problem}{Problem}
\newtheorem{proposition}{Proposition}[section]
\newtheorem{lemma}{Lemma}[section]
\newtheorem{corollary}[theorem]{Corollary}
\newtheorem{example}{Example}[section]
\newtheorem{definition}[problem]{Definition}
\newcommand{\BEQA}{\begin{eqnarray}}
\newcommand{\EEQA}{\end{eqnarray}}
\newcommand{\define}{\stackrel{\triangle}{=}}
\newcommand{\brak}[1]{\langle #1 \rangle}
\theoremstyle{remark}
\newtheorem{rem}{Remark}

\begin{document}
\bibliographystyle{IEEEtran}
\vspace{3cm}
\title{\textbf{GATE 2022 EE}}
\author{EE23BTECH11023-ABHIGNYA GOGULA}
\maketitle
\newpage
\bigskip
\renewcommand{\thefigure}{\theenumi}
\renewcommand{\thetable}{\theenumi}
\textbf{Question27:}
\\An inductor having a $Q$-factor of 60 is connected in series with a capacitor having a $Q$-factor of 240. The overall $Q$-factor of the circuit is \_\_\_\_\_\_\_\_\_\_. (Round off to the nearest integer) \\
\hfill Gate 2022 EE Question 27\\
\section*{Solution}
\fi
\begin{circuitikz}
    \draw (0,0) to[R, l=$R_1$] (2,0) to[L, l=$L$] (4,0);
\end{circuitikz}
\begin{align}
Q_1=\frac{\omega_0 L}{R_1}
\end{align}
\begin{circuitikz}
    \draw (0,0) to[R, l=$R_2$] (2,0) to[C, l=$C$] (4,0);
\end{circuitikz}
\begin{align}
Q_2=\frac{1}{\omega_0 C R_2}
\end{align}
at resonance as $\omega_0 L =\frac{1}{\omega_0 C}$ hence
\begin{align}
Q_2=\frac{\omega_0 L}{R_2}
\end{align}
\begin{circuitikz}
    \draw (0,0) to[R, l=$R_1$] (2,0) to[L, l=$L$] (4,0) to[R, l=$R_2$] (6,0) to[C, l=$C$] (8,0);
\end{circuitikz}
\begin{align}
Q = \frac{\omega_0 L}{R_1+R_2}\\
Q = \frac{1}{\frac{R_1}{\omega_0 L}+\frac{R_2}{\omega_0 L}}
\end{align}
\begin{equation}
Q =\frac{Q_1 Q_2}{Q_1+Q_2}
\label{eq:EE 27eq1}
\end{equation}
then from \eqref{eq:EE 27eq1}
\begin{align}
Q=\frac{60 \times 240}{60+240}\\
Q=48
\end{align}
%\end{document}
